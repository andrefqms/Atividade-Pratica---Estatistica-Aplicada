\documentclass[]{article}
\usepackage{lmodern}
\usepackage{amssymb,amsmath}
\usepackage{ifxetex,ifluatex}
\usepackage{fixltx2e} % provides \textsubscript
\ifnum 0\ifxetex 1\fi\ifluatex 1\fi=0 % if pdftex
  \usepackage[T1]{fontenc}
  \usepackage[utf8]{inputenc}
\else % if luatex or xelatex
  \ifxetex
    \usepackage{mathspec}
  \else
    \usepackage{fontspec}
  \fi
  \defaultfontfeatures{Ligatures=TeX,Scale=MatchLowercase}
\fi
% use upquote if available, for straight quotes in verbatim environments
\IfFileExists{upquote.sty}{\usepackage{upquote}}{}
% use microtype if available
\IfFileExists{microtype.sty}{%
\usepackage{microtype}
\UseMicrotypeSet[protrusion]{basicmath} % disable protrusion for tt fonts
}{}
\usepackage[margin=1in]{geometry}
\usepackage{hyperref}
\hypersetup{unicode=true,
            pdftitle={Atividade Pratica 3 de Estatistica Aplicada},
            pdfauthor={André Filipe Queiroz de Melo e Soares},
            pdfborder={0 0 0},
            breaklinks=true}
\urlstyle{same}  % don't use monospace font for urls
\usepackage{color}
\usepackage{fancyvrb}
\newcommand{\VerbBar}{|}
\newcommand{\VERB}{\Verb[commandchars=\\\{\}]}
\DefineVerbatimEnvironment{Highlighting}{Verbatim}{commandchars=\\\{\}}
% Add ',fontsize=\small' for more characters per line
\usepackage{framed}
\definecolor{shadecolor}{RGB}{248,248,248}
\newenvironment{Shaded}{\begin{snugshade}}{\end{snugshade}}
\newcommand{\KeywordTok}[1]{\textcolor[rgb]{0.13,0.29,0.53}{\textbf{#1}}}
\newcommand{\DataTypeTok}[1]{\textcolor[rgb]{0.13,0.29,0.53}{#1}}
\newcommand{\DecValTok}[1]{\textcolor[rgb]{0.00,0.00,0.81}{#1}}
\newcommand{\BaseNTok}[1]{\textcolor[rgb]{0.00,0.00,0.81}{#1}}
\newcommand{\FloatTok}[1]{\textcolor[rgb]{0.00,0.00,0.81}{#1}}
\newcommand{\ConstantTok}[1]{\textcolor[rgb]{0.00,0.00,0.00}{#1}}
\newcommand{\CharTok}[1]{\textcolor[rgb]{0.31,0.60,0.02}{#1}}
\newcommand{\SpecialCharTok}[1]{\textcolor[rgb]{0.00,0.00,0.00}{#1}}
\newcommand{\StringTok}[1]{\textcolor[rgb]{0.31,0.60,0.02}{#1}}
\newcommand{\VerbatimStringTok}[1]{\textcolor[rgb]{0.31,0.60,0.02}{#1}}
\newcommand{\SpecialStringTok}[1]{\textcolor[rgb]{0.31,0.60,0.02}{#1}}
\newcommand{\ImportTok}[1]{#1}
\newcommand{\CommentTok}[1]{\textcolor[rgb]{0.56,0.35,0.01}{\textit{#1}}}
\newcommand{\DocumentationTok}[1]{\textcolor[rgb]{0.56,0.35,0.01}{\textbf{\textit{#1}}}}
\newcommand{\AnnotationTok}[1]{\textcolor[rgb]{0.56,0.35,0.01}{\textbf{\textit{#1}}}}
\newcommand{\CommentVarTok}[1]{\textcolor[rgb]{0.56,0.35,0.01}{\textbf{\textit{#1}}}}
\newcommand{\OtherTok}[1]{\textcolor[rgb]{0.56,0.35,0.01}{#1}}
\newcommand{\FunctionTok}[1]{\textcolor[rgb]{0.00,0.00,0.00}{#1}}
\newcommand{\VariableTok}[1]{\textcolor[rgb]{0.00,0.00,0.00}{#1}}
\newcommand{\ControlFlowTok}[1]{\textcolor[rgb]{0.13,0.29,0.53}{\textbf{#1}}}
\newcommand{\OperatorTok}[1]{\textcolor[rgb]{0.81,0.36,0.00}{\textbf{#1}}}
\newcommand{\BuiltInTok}[1]{#1}
\newcommand{\ExtensionTok}[1]{#1}
\newcommand{\PreprocessorTok}[1]{\textcolor[rgb]{0.56,0.35,0.01}{\textit{#1}}}
\newcommand{\AttributeTok}[1]{\textcolor[rgb]{0.77,0.63,0.00}{#1}}
\newcommand{\RegionMarkerTok}[1]{#1}
\newcommand{\InformationTok}[1]{\textcolor[rgb]{0.56,0.35,0.01}{\textbf{\textit{#1}}}}
\newcommand{\WarningTok}[1]{\textcolor[rgb]{0.56,0.35,0.01}{\textbf{\textit{#1}}}}
\newcommand{\AlertTok}[1]{\textcolor[rgb]{0.94,0.16,0.16}{#1}}
\newcommand{\ErrorTok}[1]{\textcolor[rgb]{0.64,0.00,0.00}{\textbf{#1}}}
\newcommand{\NormalTok}[1]{#1}
\usepackage{graphicx,grffile}
\makeatletter
\def\maxwidth{\ifdim\Gin@nat@width>\linewidth\linewidth\else\Gin@nat@width\fi}
\def\maxheight{\ifdim\Gin@nat@height>\textheight\textheight\else\Gin@nat@height\fi}
\makeatother
% Scale images if necessary, so that they will not overflow the page
% margins by default, and it is still possible to overwrite the defaults
% using explicit options in \includegraphics[width, height, ...]{}
\setkeys{Gin}{width=\maxwidth,height=\maxheight,keepaspectratio}
\IfFileExists{parskip.sty}{%
\usepackage{parskip}
}{% else
\setlength{\parindent}{0pt}
\setlength{\parskip}{6pt plus 2pt minus 1pt}
}
\setlength{\emergencystretch}{3em}  % prevent overfull lines
\providecommand{\tightlist}{%
  \setlength{\itemsep}{0pt}\setlength{\parskip}{0pt}}
\setcounter{secnumdepth}{0}
% Redefines (sub)paragraphs to behave more like sections
\ifx\paragraph\undefined\else
\let\oldparagraph\paragraph
\renewcommand{\paragraph}[1]{\oldparagraph{#1}\mbox{}}
\fi
\ifx\subparagraph\undefined\else
\let\oldsubparagraph\subparagraph
\renewcommand{\subparagraph}[1]{\oldsubparagraph{#1}\mbox{}}
\fi

%%% Use protect on footnotes to avoid problems with footnotes in titles
\let\rmarkdownfootnote\footnote%
\def\footnote{\protect\rmarkdownfootnote}

%%% Change title format to be more compact
\usepackage{titling}

% Create subtitle command for use in maketitle
\newcommand{\subtitle}[1]{
  \posttitle{
    \begin{center}\large#1\end{center}
    }
}

\setlength{\droptitle}{-2em}

  \title{Atividade Pratica 3 de Estatistica Aplicada}
    \pretitle{\vspace{\droptitle}\centering\huge}
  \posttitle{\par}
    \author{André Filipe Queiroz de Melo e Soares}
    \preauthor{\centering\large\emph}
  \postauthor{\par}
      \predate{\centering\large\emph}
  \postdate{\par}
    \date{12 de Setembro, 2018}


\begin{document}
\maketitle

\subsection{QUESTAO 1}\label{questao-1}

\begin{Shaded}
\begin{Highlighting}[]
\CommentTok{# LETRA A }


\NormalTok{med10 =}\StringTok{ }\KeywordTok{vector}\NormalTok{()}
\NormalTok{varia10 =}\StringTok{ }\KeywordTok{vector}\NormalTok{()}

\NormalTok{med30 =}\StringTok{ }\KeywordTok{vector}\NormalTok{()}
\NormalTok{varia30 =}\StringTok{ }\KeywordTok{vector}\NormalTok{()}

\NormalTok{med50 =}\StringTok{ }\KeywordTok{vector}\NormalTok{()}
\NormalTok{varia50 =}\StringTok{ }\KeywordTok{vector}\NormalTok{()}

\ControlFlowTok{for}\NormalTok{ (i }\ControlFlowTok{in} \DecValTok{1}\OperatorTok{:}\DecValTok{1000}\NormalTok{) \{}
\NormalTok{  populacao10 =}\StringTok{ }\KeywordTok{sample}\NormalTok{(}\DecValTok{1000}\NormalTok{, }\DecValTok{10}\NormalTok{, }\OtherTok{TRUE}\NormalTok{)}
\NormalTok{  populacao30 =}\StringTok{ }\KeywordTok{sample}\NormalTok{(}\DecValTok{1000}\NormalTok{, }\DecValTok{30}\NormalTok{, }\OtherTok{TRUE}\NormalTok{)}
\NormalTok{  populacao50 =}\StringTok{ }\KeywordTok{sample}\NormalTok{(}\DecValTok{1000}\NormalTok{, }\DecValTok{50}\NormalTok{, }\OtherTok{TRUE}\NormalTok{)}
  
\NormalTok{  varia10[i] =}\StringTok{ }\KeywordTok{var}\NormalTok{(populacao10)}
\NormalTok{  med10[i] =}\StringTok{ }\KeywordTok{mean}\NormalTok{(populacao10)}
  
\NormalTok{  med30[i] =}\StringTok{ }\KeywordTok{mean}\NormalTok{(populacao30)}
\NormalTok{  varia30[i] =}\StringTok{ }\KeywordTok{var}\NormalTok{(populacao30)}
  
\NormalTok{  varia50[i] =}\StringTok{ }\KeywordTok{var}\NormalTok{(populacao50)}
\NormalTok{  med50[i] =}\StringTok{ }\KeywordTok{mean}\NormalTok{(populacao50)}
  
\NormalTok{\}}


\KeywordTok{hist}\NormalTok{(med10, }\DataTypeTok{main =} \StringTok{"Histograma da média de amostras de tamanho 10 que tem reposição"}\NormalTok{, }\DataTypeTok{border =} \StringTok{"white"}\NormalTok{, }\DataTypeTok{col =} \StringTok{"red"}\NormalTok{)}
\end{Highlighting}
\end{Shaded}

\includegraphics{pratica3_files/figure-latex/unnamed-chunk-1-1.pdf}

\begin{Shaded}
\begin{Highlighting}[]
\KeywordTok{hist}\NormalTok{(varia10, }\DataTypeTok{main =} \StringTok{"Histograma da variância das amostras de tamanho 10 que tem reposição"}\NormalTok{, }\DataTypeTok{border =} \StringTok{"white"}\NormalTok{, }\DataTypeTok{col =} \StringTok{"orange"}\NormalTok{)}
\end{Highlighting}
\end{Shaded}

\includegraphics{pratica3_files/figure-latex/unnamed-chunk-1-2.pdf}

\begin{Shaded}
\begin{Highlighting}[]
\KeywordTok{hist}\NormalTok{(med30, }\DataTypeTok{main =} \StringTok{"Histograma da média de amostras de tamanho 30 que tem reposição"}\NormalTok{, }\DataTypeTok{border =} \StringTok{"white"}\NormalTok{, }\DataTypeTok{col =} \StringTok{"light blue"}\NormalTok{)}
\end{Highlighting}
\end{Shaded}

\includegraphics{pratica3_files/figure-latex/unnamed-chunk-1-3.pdf}

\begin{Shaded}
\begin{Highlighting}[]
\KeywordTok{hist}\NormalTok{(varia30, }\DataTypeTok{main =} \StringTok{"Histograma da variância de amostras de tamanho 30 que tem reposição"}\NormalTok{, }\DataTypeTok{border =} \StringTok{"white"}\NormalTok{, }\DataTypeTok{col =} \StringTok{"dark red"}\NormalTok{)}
\end{Highlighting}
\end{Shaded}

\includegraphics{pratica3_files/figure-latex/unnamed-chunk-1-4.pdf}

\begin{Shaded}
\begin{Highlighting}[]
\KeywordTok{hist}\NormalTok{(med50, }\DataTypeTok{main =} \StringTok{"Histograma da média das amostras de tamanho 50 com reposição"}\NormalTok{, }\DataTypeTok{border =} \StringTok{"white"}\NormalTok{, }\DataTypeTok{col =} \StringTok{"dark blue"}\NormalTok{)}
\end{Highlighting}
\end{Shaded}

\includegraphics{pratica3_files/figure-latex/unnamed-chunk-1-5.pdf}

\begin{Shaded}
\begin{Highlighting}[]
\KeywordTok{hist}\NormalTok{(varia50, }\DataTypeTok{main =} \StringTok{"Histograma da variância de amostras de tamanho 50 que tem reposição"}\NormalTok{, }\DataTypeTok{border =} \StringTok{"white"}\NormalTok{, }\DataTypeTok{col =} \StringTok{"purple"}\NormalTok{)}
\end{Highlighting}
\end{Shaded}

\includegraphics{pratica3_files/figure-latex/unnamed-chunk-1-6.pdf}

\begin{Shaded}
\begin{Highlighting}[]
\CommentTok{# LETRA B}


\NormalTok{media10 =}\StringTok{ }\KeywordTok{vector}\NormalTok{()}
\NormalTok{var10 =}\StringTok{ }\KeywordTok{vector}\NormalTok{()}

\NormalTok{media30 =}\StringTok{ }\KeywordTok{vector}\NormalTok{()}
\NormalTok{var30 =}\StringTok{ }\KeywordTok{vector}\NormalTok{()}

\NormalTok{media50 =}\StringTok{ }\KeywordTok{vector}\NormalTok{()}
\NormalTok{var50 =}\StringTok{ }\KeywordTok{vector}\NormalTok{()}

\ControlFlowTok{for}\NormalTok{ (i }\ControlFlowTok{in} \DecValTok{1}\OperatorTok{:}\DecValTok{1000}\NormalTok{) \{}
\NormalTok{  populacao10 =}\StringTok{ }\KeywordTok{sample}\NormalTok{(}\DecValTok{1000}\NormalTok{, }\DecValTok{10}\NormalTok{, }\OtherTok{FALSE}\NormalTok{)}
\NormalTok{  populacao30 =}\StringTok{ }\KeywordTok{sample}\NormalTok{(}\DecValTok{1000}\NormalTok{, }\DecValTok{30}\NormalTok{, }\OtherTok{FALSE}\NormalTok{)}
\NormalTok{  populacao50 =}\StringTok{ }\KeywordTok{sample}\NormalTok{(}\DecValTok{1000}\NormalTok{, }\DecValTok{50}\NormalTok{, }\OtherTok{FALSE}\NormalTok{)}
  
\NormalTok{  media30[i] =}\StringTok{ }\KeywordTok{mean}\NormalTok{(populacao30)}
\NormalTok{  var30[i] =}\StringTok{ }\KeywordTok{var}\NormalTok{(populacao30)}
  
\NormalTok{  media10[i] =}\StringTok{ }\KeywordTok{mean}\NormalTok{(populacao10)}
\NormalTok{  var10[i] =}\StringTok{ }\KeywordTok{var}\NormalTok{(populacao10)}
  
\NormalTok{  var50[i] =}\StringTok{ }\KeywordTok{var}\NormalTok{(populacao50)}
\NormalTok{  media50[i] =}\StringTok{ }\KeywordTok{mean}\NormalTok{(populacao50)}
  
\NormalTok{\}}


\KeywordTok{hist}\NormalTok{(media10, }\DataTypeTok{main =} \StringTok{"Histograma da média das amostras de tamanho 10 sem repor"}\NormalTok{, }\DataTypeTok{border =} \StringTok{"white"}\NormalTok{, }\DataTypeTok{col =} \StringTok{"green"}\NormalTok{)}
\end{Highlighting}
\end{Shaded}

\includegraphics{pratica3_files/figure-latex/unnamed-chunk-1-7.pdf}

\begin{Shaded}
\begin{Highlighting}[]
\KeywordTok{hist}\NormalTok{(var10, }\DataTypeTok{main =} \StringTok{"Histograma da variância das amostras de tamanho 10 sem reposição"}\NormalTok{, }\DataTypeTok{border =} \StringTok{"white"}\NormalTok{, }\DataTypeTok{col =} \StringTok{"blue"}\NormalTok{)}
\end{Highlighting}
\end{Shaded}

\includegraphics{pratica3_files/figure-latex/unnamed-chunk-1-8.pdf}

\begin{Shaded}
\begin{Highlighting}[]
\KeywordTok{hist}\NormalTok{(media30, }\DataTypeTok{main =} \StringTok{"Histograma da média das amostras de tamanho 30 sem repor"}\NormalTok{, }\DataTypeTok{border =} \StringTok{"white"}\NormalTok{, }\DataTypeTok{col =} \StringTok{"gray"}\NormalTok{)}
\end{Highlighting}
\end{Shaded}

\includegraphics{pratica3_files/figure-latex/unnamed-chunk-1-9.pdf}

\begin{Shaded}
\begin{Highlighting}[]
\KeywordTok{hist}\NormalTok{(var30, }\DataTypeTok{main =} \StringTok{"Histograma da variância das amostras de tamanho 30 sem reposição"}\NormalTok{, }\DataTypeTok{border =} \StringTok{"white"}\NormalTok{, }\DataTypeTok{col =} \StringTok{"dark green"}\NormalTok{)}
\end{Highlighting}
\end{Shaded}

\includegraphics{pratica3_files/figure-latex/unnamed-chunk-1-10.pdf}

\begin{Shaded}
\begin{Highlighting}[]
\KeywordTok{hist}\NormalTok{(media50, }\DataTypeTok{main =} \StringTok{"Histograma da média das amostras de tamanho 50 sem repor"}\NormalTok{, }\DataTypeTok{border =} \StringTok{"white"}\NormalTok{, }\DataTypeTok{col =} \StringTok{"brown"}\NormalTok{)}
\end{Highlighting}
\end{Shaded}

\includegraphics{pratica3_files/figure-latex/unnamed-chunk-1-11.pdf}

\begin{Shaded}
\begin{Highlighting}[]
\KeywordTok{hist}\NormalTok{(var50, }\DataTypeTok{main =} \StringTok{"Histograma da variância das amostras de tamanho 50 sem repor"}\NormalTok{, }\DataTypeTok{border =} \StringTok{"white"}\NormalTok{, }\DataTypeTok{col =} \StringTok{"pink"}\NormalTok{)}
\end{Highlighting}
\end{Shaded}

\includegraphics{pratica3_files/figure-latex/unnamed-chunk-1-12.pdf}

\begin{Shaded}
\begin{Highlighting}[]
\CommentTok{# LETRA C}

 \CommentTok{#Média respectivamente da população de 10, amostras de 10 com reposicao e sem reposicao}

\KeywordTok{mean}\NormalTok{(populacao10)}
\end{Highlighting}
\end{Shaded}

\begin{verbatim}
## [1] 599.5
\end{verbatim}

\begin{Shaded}
\begin{Highlighting}[]
\KeywordTok{mean}\NormalTok{(med10)}
\end{Highlighting}
\end{Shaded}

\begin{verbatim}
## [1] 499.4052
\end{verbatim}

\begin{Shaded}
\begin{Highlighting}[]
\KeywordTok{mean}\NormalTok{(media10)}
\end{Highlighting}
\end{Shaded}

\begin{verbatim}
## [1] 497.4144
\end{verbatim}

\begin{Shaded}
\begin{Highlighting}[]
 \CommentTok{#Média respectivamente da população de 10, amostras de 10 sem reposicao e com reposicao}

\KeywordTok{var}\NormalTok{(populacao10)}
\end{Highlighting}
\end{Shaded}

\begin{verbatim}
## [1] 118562.1
\end{verbatim}

\begin{Shaded}
\begin{Highlighting}[]
\KeywordTok{var}\NormalTok{(var10)}
\end{Highlighting}
\end{Shaded}

\begin{verbatim}
## [1] 671484305
\end{verbatim}

\begin{Shaded}
\begin{Highlighting}[]
\KeywordTok{var}\NormalTok{(varia10)}
\end{Highlighting}
\end{Shaded}

\begin{verbatim}
## [1] 780924144
\end{verbatim}

\begin{Shaded}
\begin{Highlighting}[]
 \CommentTok{#Média respectivamente da população de 30, amostras de 30 com reposicao e sem reposicao}

\KeywordTok{mean}\NormalTok{(populacao30)}
\end{Highlighting}
\end{Shaded}

\begin{verbatim}
## [1] 484.9
\end{verbatim}

\begin{Shaded}
\begin{Highlighting}[]
\KeywordTok{mean}\NormalTok{(med30)}
\end{Highlighting}
\end{Shaded}

\begin{verbatim}
## [1] 501.2585
\end{verbatim}

\begin{Shaded}
\begin{Highlighting}[]
\KeywordTok{mean}\NormalTok{(media30)}
\end{Highlighting}
\end{Shaded}

\begin{verbatim}
## [1] 497.7047
\end{verbatim}

\begin{Shaded}
\begin{Highlighting}[]
 \CommentTok{#Variância respectivamente da população de 30, amostras de 30 sem reposicao e com reposicao}

\KeywordTok{var}\NormalTok{(populacao30)}
\end{Highlighting}
\end{Shaded}

\begin{verbatim}
## [1] 59607.47
\end{verbatim}

\begin{Shaded}
\begin{Highlighting}[]
\KeywordTok{var}\NormalTok{(var30)}
\end{Highlighting}
\end{Shaded}

\begin{verbatim}
## [1] 190635514
\end{verbatim}

\begin{Shaded}
\begin{Highlighting}[]
\KeywordTok{var}\NormalTok{(varia30)}
\end{Highlighting}
\end{Shaded}

\begin{verbatim}
## [1] 205425205
\end{verbatim}

\begin{Shaded}
\begin{Highlighting}[]
 \CommentTok{#Média respectivamente da população de 50, amostras de 50 com reposicao e sem reposicao}

\KeywordTok{mean}\NormalTok{(populacao50)}
\end{Highlighting}
\end{Shaded}

\begin{verbatim}
## [1] 450.42
\end{verbatim}

\begin{Shaded}
\begin{Highlighting}[]
\KeywordTok{mean}\NormalTok{(med50)}
\end{Highlighting}
\end{Shaded}

\begin{verbatim}
## [1] 499.9874
\end{verbatim}

\begin{Shaded}
\begin{Highlighting}[]
\KeywordTok{mean}\NormalTok{(media50)}
\end{Highlighting}
\end{Shaded}

\begin{verbatim}
## [1] 500.8577
\end{verbatim}

\begin{Shaded}
\begin{Highlighting}[]
 \CommentTok{#Variância respectivamente da população de 50, amostras de 50 sem reposicao e com reposicao}

\KeywordTok{var}\NormalTok{(populacao50)}
\end{Highlighting}
\end{Shaded}

\begin{verbatim}
## [1] 85480.82
\end{verbatim}

\begin{Shaded}
\begin{Highlighting}[]
\KeywordTok{var}\NormalTok{(var50)}
\end{Highlighting}
\end{Shaded}

\begin{verbatim}
## [1] 114536714
\end{verbatim}

\begin{Shaded}
\begin{Highlighting}[]
\KeywordTok{var}\NormalTok{(varia50)}
\end{Highlighting}
\end{Shaded}

\begin{verbatim}
## [1] 121577170
\end{verbatim}

\subsection{QUESTAO 2}\label{questao-2}

\begin{Shaded}
\begin{Highlighting}[]
\NormalTok{media =}\StringTok{ }\DecValTok{0}
\NormalTok{variancias =}\StringTok{ }\DecValTok{0}
\ControlFlowTok{for}\NormalTok{ (i }\ControlFlowTok{in} \DecValTok{1}\OperatorTok{:}\DecValTok{1000}\NormalTok{) \{}
\NormalTok{  dist =}\StringTok{ }\KeywordTok{rpois}\NormalTok{(}\DecValTok{100}\NormalTok{,}\DecValTok{10}\NormalTok{)}
\NormalTok{  media[i] =}\StringTok{ }\KeywordTok{mean}\NormalTok{(dist)}
\NormalTok{  variancias[i] =}\StringTok{ }\KeywordTok{var}\NormalTok{(dist)}
\NormalTok{\}}
\KeywordTok{hist}\NormalTok{(media,}\DataTypeTok{main =} \StringTok{"Histograma das médias"}\NormalTok{,}\DataTypeTok{col =} \StringTok{"black"}\NormalTok{, }\DataTypeTok{border =} \StringTok{"white"}\NormalTok{)}
\end{Highlighting}
\end{Shaded}

\includegraphics{pratica3_files/figure-latex/unnamed-chunk-2-1.pdf}

\begin{Shaded}
\begin{Highlighting}[]
\KeywordTok{hist}\NormalTok{(variancias,}\DataTypeTok{main =} \StringTok{"Histogramas das variâncias"}\NormalTok{, }\DataTypeTok{col =} \StringTok{"dark gray"}\NormalTok{,}\DataTypeTok{border =} \StringTok{"white"}\NormalTok{)}
\end{Highlighting}
\end{Shaded}

\includegraphics{pratica3_files/figure-latex/unnamed-chunk-2-2.pdf}

\subsection{QUESTAO 3}\label{questao-3}

\begin{Shaded}
\begin{Highlighting}[]
\NormalTok{exp =}\StringTok{ }\DecValTok{0}
\ControlFlowTok{for}\NormalTok{ (i }\ControlFlowTok{in} \DecValTok{1}\OperatorTok{:}\DecValTok{1000}\NormalTok{) \{}
\NormalTok{  distribuicao =}\StringTok{ }\KeywordTok{rexp}\NormalTok{(}\DecValTok{100}\NormalTok{,}\DecValTok{5}\NormalTok{)}
\NormalTok{  T =}\StringTok{ }\DecValTok{1}\OperatorTok{/}\KeywordTok{mean}\NormalTok{(distribuicao)}
\NormalTok{  exp[i] =}\StringTok{ }\NormalTok{T}
\NormalTok{\}}

\KeywordTok{hist}\NormalTok{(exp,}\DataTypeTok{col =} \StringTok{"dark green"}\NormalTok{, }\DataTypeTok{border =} \StringTok{"white"}\NormalTok{,}\DataTypeTok{main =}\StringTok{"Histograma da distribuição"}\NormalTok{)}
\end{Highlighting}
\end{Shaded}

\includegraphics{pratica3_files/figure-latex/unnamed-chunk-3-1.pdf}


\end{document}
